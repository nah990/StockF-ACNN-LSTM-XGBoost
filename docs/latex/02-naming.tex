\section*{ОПРЕДЕЛЕНИЯ, ОБОЗНАЧЕНИЯ И СОКРАЩЕНИЯ}

\begin{enumerate}[leftmargin=1.6\parindent]

\item ARIMA (от англ. Autoregressive integrated moving average) --- интегрированная модель авторегрессии —-- скользящего среднего ---  модель и методология анализа временных рядов.
\item CNN (от англ. Convolutional neural network) --- сверточная нейронная сеть. 
\item ACNN (от англ. Attention Convolutional Neural Network) --- модель сверточной нейронной сети на основе механизма внимания.
\item RNN (от англ. Recurrent neural network) --- рекуррентная нейронная сеть. 
\item LSTM (от англ. long-short Term Memory) --- разновидность архитектуры рекуррентных нейронных сетей, способная запоминать значения как на короткие, так и на длинные промежутки времени. 
\item BiLSTM (от англ. Bidirectional long-short Term Memory) —-- это модель обработки последовательности, состоящая из двух LSTM: одна принимает входные данные в прямом направлении, а другая — в обратном.
\item Seq2Seq (от англ. Sequence to sequence) ---  семейство подходов машинного обучения, состоящее из двух рекуррентных сетей: кодировщика и декодировщика.
\item ADF тест (от англ. Augmented Dickey–Fuller test) --- методика анализа временных рядов для проверки на стационарность.
\item FFNN (от англ. feed-forward neural network) --- нейронная сеть, в которой соединения между узлами не образуют цикл.
\item HMM (от англ. Hidden Markov Model) --- модель, имитирующая работу процесса, похожего на марковский процесс с неизвестными параметрами, и задачей ставится разгадывание неизвестных параметров на основе наблюдаемых.

\end{enumerate}
\pagebreak