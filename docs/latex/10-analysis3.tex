\subsection{Анализ существующих моделей прогнозирования временных рядов}

\subsubsection{Классификация моделей}

\par Модели временных рядов —-- математические модели прогнозирования, которые стремятся найти зависимость будущего значения от прошлого внутри самого процесса и на этой зависимости вычислить прогноз. Эти модели универсальны для различных предметных областей, то есть их общий вид не меняется в зависимости от природы временного ряда. Перед началом обзора моделей стоит отметить тот факт, что названия моделей и соответствующих им методов зачастуют совпадают, также принято использовать английские аббревиатуры \cite{math-model}. Сами методы прогнозирования как правило бывают двух видов:
\begin{itemize}[leftmargin=1.6\parindent]
	\item[---] \textit{Интуитивные методы}. Имеют дело с индивидуальными суждениями экспертов. Используются и применяются тогда, когда объект прогнозоварования либо слишком прост, либо, напротив, настолько сложен, что аналитически учесть влияние внешних факторов невозможно;
	\item[---] \textit{Формализованные методы}. Имеют дело с математическими моделями прогнозирования, то есть определяют такую зависимость, которая позволила бы вычислить будущее значение процесса. В следующей работе \cite{math-model-1} модели разделяются на статистические и структурные.
\end{itemize}

\par \textbf{В статистических моделях} функциональная зависимость между будущими и фактическими значениями временного ряда, а также внешними факторами задана аналитически. К статистическим моделям относятся следующие группы:

\begin{itemize}[leftmargin=1.6\parindent]
	\item[---] Регрессионные модели;
	\item[---] Авторегрессионные модели;
	\item[---] Модели экспоненциального сглаживания
\end{itemize}

\par \textbf{В структурных моделях} функциональная зависимость между будущими и фактическими значениями временного ряда, а также внешними факторами задана структурно. К структурным моделям относятся следующие группы:

\begin{itemize}[leftmargin=1.6\parindent]
	\item[---] Нейросетевые модели;
	\item[---] Модели на базе цепей Маркова;
	\item[---] Модели на базе классификационно-регрессионных деревьев.
\end{itemize}

\subsubsection{Концепция комбинированной модели}

\par В рамках данной работы рассматривается комбинированная модель, предполагающая возможность компенсирования недостатков одних моделей при помощи других. Раннее изложенный вывод о существующих решениях подводит к тому, что основой комбинированной модели должна выступать как и статическая модель, так и структурная. На основе данных работ \cite{math-model,math-model-1,math-model-2,math-model-3} есть все основания утверждать, что среди статистических моделей наиболее подходящей является autoregressive integrated moving average (ARIMA). В свою очередь среди структурных моделей, а именно нейросетевых, неплохие результаты демонстрирует long-short term memory (LSTM), являющаяся по своей сути разновидностью архитектуры рекуррентных нейронных сетей \cite{LSTM-state,framework,ml-vakh}, которая в комбинации со сверточной нейронной сетью на основе механизма внимания добивается весьма достоверных результатов \cite{CNN-LSTM,CNN-model,combining,short-term}.

\subsection{Выводы из аналитического раздела}

\par Была произведена классификация существующих методов прогнозирования временных рядов на фондовом рынке. Были решены следующие задачи:

\begin{itemize}[leftmargin=1.6\parindent]
    \item[---] Описаны основные понятия предметной области и обозначена проблема;
	\item[---] Проведен анализ существующих методов и средств прогнозирования временных рядов;
	\item[---] Отобраны наиболее удовлетворяющие параметрам эффективности методы.
\end{itemize}

\pagebreak