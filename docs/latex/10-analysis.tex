\section{Аналитический раздел}

\par В данном разделе рассматривается задача прогнозирования временных рядов на фондовом рынке. Происходит постановка задачи, описываются основные понятия предметной области. Дается анализ существующих методов и средств прогнозирования временных рядов с дальнейшим отбором наиболее эффективных для комбинирования. Формулируется концепция комбинированного метода на уровне идеи, происходит формализация задачи. 


\subsection{Анализ предметной области}

\subsubsection{Постановка задачи}



 \par С того момента как Россия вступила на рыночный путь развития –
 рынок ценных бумаг занял лидирующее место среди институтов новой
 экономики. В наше время наблюдается активный рост количества частных инвесторов на Московской бирже \cite{rus-invest}. Причина данного явления кроется в главном свойстве ценной бумаги --- это способность приносить доход. Также особую привлекательность добавляет тот факт, что грамотное распределение денежных средств на фондовом рынке потенциально обладает большей прибыльностью, нежели банковский вклад.
 
 \par Совершенно очевидно, что различного рода анализ финансовых временных рядов с целью прогнозирования непосредственном образом превращается в прибыль. Тем не менее важно понимать, что бурный рост финансовых рынков сопровождается соответсвующим ростом порога вхождения. Подтвержденим этого служит статистика о том, что большая часть инвесторов, совершающих операции на фондовом рынке имеет достаточно низкий уровень общей прибыльности операций, некоторые же терпят значительные убытки \cite{tinkoff-stat}\cite{most-stat}. Одной из возможных причин этому служит необходимость конкуриривания рядового игрока с нарастающей автоматизацией отрасли биржевой торговли с эффективным инструментарием \cite{autopilot}. 
 
 \par Финансовые боты задействованы в основном в сверхскоростных сделках, совершаемых за долю секунды, именуемыми скальпингом. Исходя из этого факта и заключения статьи \cite{scalping}, наиболее выигрышным решением сложившейся ситуации представляется использование инструментов прогнозирования на более длительном отрезке времени. 
 
 \par Также стоит отметить, что прогнозирование котировок акций –-- комплексная и сложная задача, представляющая собой анализ огромного количества факторов, которые невозможно учесть в полном объеме. Ситуация дополнительно усугубляется появлением человеческого фактора, неизбежно ведущим к стохастичности свойств и неопределенности экономической системы. Причиной этому служит крайне сложная структура модели поведения человека, которая зачастую может быть иррациональной. \cite{behavioral-patterns} Определяет ее, как правило доступные информационные ресурсы. \cite{news-impact} При чем вовсе не обязательно данные ресурсы соответствует действительности. \cite{news-fake}
 
 \par На основание озвученной проблемы возникает потребность в классификации существующих методов прогнозирования временных рядов на фондовом рынке, которые станут основой для разработки комбинированного метода. 
 
\subsubsection{Основные понятия}





\textbf{Классические методы анализа фондового рынка}

\par Существует три основных метода анализа фондового рынка: фундаментальный анализ, технический анализ и интуитивный подход к анализу.\cite{fund-anal}

\par Фундаментальный анализ —-- анализ финансовой
деятельности компании с целью адекватной оценки ценных бумаг,
при этом особое внимание уделяется будущим доходам компании,
ожидаемым дивидендам и будущим процентным ставкам, а так
же оценка риска деятельности компании.\cite{fund-anal}

\par Технический анализ --- основан на следующей гипотезе, утверждающей, что рыночные цены учитывают все знания, желания и действия всех участников рынка, отражая их в своей динамике. В результате и цена, и объем включают в себя каждую сделку, совершенную многотысячной армией трейдеров. Исходя из этой гипотезы существует значительное количество индикаторов для определения моментов покупки и продажи ценных бумаг. \cite{fund-anal}

\par Фундаментальный и технический анализ – оба направления по сути основываются на статистических данных о рынках, но находятся на противоположенных сторонах. Первый дает возможность
разработки долгосрочной стратегии, а второй —-- только краткосрочной.

\par Интуитивный (психологический) подход к анализу ---  наимение рациональный среди перечисленных, зачастую не причисляется к классическим методам. Как правило, не приводит к долговременному успеху ввиду его нестабильности. Применение оправдано только при наличие обширного опыта, в сочетание с необходимостью быстрой реакции на стремительно изменяющиеся тенденции финансового рынка.

\par На практике рекомендуется использовать элементы всех 3 видов
анализа. Это не конкурирующие, а взаимодополняющие друг друга
инструменты.

\textbf{Финансовые временные ряды}

\par Временной ряд --– это определенный признак, значение которого отслеживается через постоянные временные интервалы. Измерения признака происходят во времени, и между разными измерениями проходит одинаковое количество времени, это является ключевой особенностью временных рядов, поскольку в случае, когда промежутки между отчетами различные, то этот процесс является случайным, следовательно, он неупорядочен во времени и целевая переменная не зависит от исторических данных. Методики анализа случайных процессов и временных рядов значительно отличаются, следовательно, важно обращать внимание на данную особенность.\cite{dynamic}

\par Задачу прогнозирования финансовых временных рядов можно определить как:
\begin{itemize}[leftmargin=1.6\parindent]
    \item[---] \textit{Классификацию}. Предсказание качественного отклика для некоторого наблюдения можно отнести к определенной категории, или классу;
	\item[---] \textit{Регрессию}. Исследование влияния одной группы непрерывных случайных величин на другую.
\end{itemize}

\par В рамках фондового рынка временные ряды, как правило, основываются на ценах или
их динамике, такие временные ряды называют финансовыми.

\par При моделировании финансовых временных рядов часто можно столкнуться со следующими проблемами:

\begin{itemize}[leftmargin=1.6\parindent]
    \item[---] \textit{Автокорреляция}. Предсказание качественного отклика для некоторого наблюдения можно отнести к определенной категории, или классу;
	\item[---] \textit{Нестационарность}. Цены акций зависят от большого количества факторов извне, характеризующихся компонентами тренда, сезонности, циклов, всплесков, которые в
    свою очередь, зависят от времени;
    \item[---] \textit{Гетероскедостичность}. Участки с низкой
волатильностью часто сменяются участками с высокой волатильностью;
    \item[---] \textit{Наличие в данных выбросов}. Выбросы представляют собой точки, которые значительно отличаются от остальных наблюдений временного ряда.
\end{itemize}
