\section*{ЗАКЛЮЧЕНИЕ}
\addcontentsline{toc}{section}{ЗАКЛЮЧЕНИЕ}

\par Трудно недооценить влияние оказываемое фондовым рынком на темпы экономического и финансового развития любой отрасли. Тем не менее эффективности инфраструктуры рынка ценных бумаг противопоставляется сложная волатильность, как следствие возможность предсказания тенденций курса акций --- это важнейшее стратегическое преимущество способное защитить, и потенциально преумножить вложения инвесторов. 
\par К сожалению, классические модели временных рядов по типу ARIMA не в состояние описать всю нелинейность прогнозирования фондового рынка, зато нейронные сети как раз обладают потенциалом на такую способность. 
\par В этой работе основное внимание было отведено комбинированной модели на базе известных моделей ARIMA, CNN-LSTM и XGBoost. Их совместная интеграция в нелинейной зависимости позволила добиться заметного улучшения точности предсказания, нежели по-одиночке. Данная модель может фиксиовать информацию о фондовом рынке за несколько периодов. Данные  предварительно обработываются через модель ARIMA, затем архитектура глубокого обучения формируется в рамках предобучения-дообучения с применение фрэймворков. Предообучение представляет собой CNN-LSTM модель на основе подхода seq2seq. Первоначально модель при помощи механизма внимания множественной свертки добывает глубинные характеристики временного ряда, затем, XGBoost регрессор применяется с целью дообучения комбинированной модели. 
\par Результаты отчетливо демонстрируют, что концепция комбинированной модели превалирует над отдельно взятыми авторегрессиоными и нейросетевыми аналогами, и как следствие является переспективной ветвью развития в вопросах прогнозированяи временных рядов на фондовом рынке.

\par В результате выполнения выпускной квалификационной работы была достигнута поставленная цель –-- исследованы существующие решения, на их основе разработан комбинированный метод прогнозирования временных рядов на фондовом рынке. На пути к достижению цели были решены следующие задачи:
\begin{itemize}[leftmargin=1.6\parindent]
    \item[---] Описаны основные понятия предметной области и обозначена проблема;
	\item[---] Проведен анализ существующих методов и средств прогнозирования временных рядов;
	\item[---] Отобраны наиболее удовлетворяющие параметрам эффективности методы.
	\item[---] Выбранные методы разработаны;
	\item[---] Разработано комбинирование реализованных методов;
\end{itemize}

\pagebreak



\pagebreak